\documentclass[12pt,a4paper,oneside]{article}

\usepackage[utf8]{inputenc}
\usepackage[portuguese]{babel}
\usepackage[T1]{fontenc}
\usepackage{amsmath}
\usepackage{amsfonts}
\usepackage{amssymb}
\usepackage{graphicx}
\usepackage{xcolor}
% Definindo novas cores
\definecolor{verde}{rgb}{0.25,0.5,0.35}

\author{\\Universidade Federal de Goiás (UFG) - Regional  Jataí\\Bacharelado em Ciência da Computação \\Teoria da Computação \\Esdras Lins Bispo Jr.}

\date{13 de julho de 2017}

\title{\sc \huge Terceiro Teste}

\begin{document}

\maketitle

{\bf ORIENTAÇÕES PARA A RESOLUÇÃO}

\small
 
\begin{itemize}
	\item A avaliação é individual, sem consulta;
	\item A pontuação máxima desta avaliação é 10,0 (dez) pontos, sendo uma das 06 (seis) componentes que formarão a média final da disciplina: quatro testes, uma prova e exercícios;
	\item A média final ($MF$) será calculada assim como se segue
	\begin{eqnarray}
		MF & = & MIN(10, S) \nonumber \\
		S & = & (\sum_{i=1}^{4} 0,2.T_i ) + 0,2.P  + 0,1.EA + EB\nonumber
	\end{eqnarray}
	em que 
	\begin{itemize}
		\item $S$ é o somatório da pontuação de todas as avaliações,
		\item $T_i$ é a pontuação obtida no teste $i$,
		\item $P$ é a pontuação obtida na prova,
		\item $EA$ é a pontuação total dos exercícios de aquecimentos, e
		\item $EB$ é a pontuação total dos exercícios-bônus.
	\end{itemize}
	\item O conteúdo exigido desta avaliação compreende o seguinte ponto apresentado no Plano de Ensino da disciplina:  (3) Problemas Decidíveis e (4) Problemas Indecidíveis.
\end{itemize}

\begin{center}
	\fbox{\large Nome: \hspace{10cm}}
\end{center}\newpage

\begin{enumerate}
	
	\section*{Terceiro Teste}
	
	\item (5,0 pt)  Seja $\mathcal{B}$ o conjunto de todas as sequências infinitas sobre $\{0,1\}$. Mostre que $\mathcal{B}$ é incontável, usando uma prova por diagonalização.
	
	\vspace{0.3cm}
	
	{\color{blue}
		R - Vamos supor, por um momento, que $\mathcal{B}$ seja contável. Sendo contável, seja $f:\mathbb{N} \rightarrow \mathcal{B}$ a suposta bijeção existente entre $\mathbb{N}$ e $\mathcal{B}$ (tendo em vista que $\mathcal{B}$ não é finito - Definição 4.14). Logo, todos os elementos de $\mathcal{B}$ deveriam participar de $f$. Entretanto, é possível construir $x \in \mathcal{B}$, a partir de $f$, que não participa da suposta bijeção.
		
		Seja $g(n,d)$ a função que retorna o $d$-ésimo dígito da sequência binária $f(n)$. Sejam $a,b \in \{0,1\} $ dois dígitos quaisquer. Definimos como $a \circ b$ a concatenação dos dígitos $a$ e $b$, nesta ordem. Também definimos como $\overline{a}$ o elemento oposto de $a$ (se $a$=0, $\overline{a} = 1$, e vice-versa) .Logo, é possível construir a sequência infinita $x$ da seguinte forma:
		\begin{center}
			$x = \overline{g(1,1)} \circ \overline{g(2,2)} \circ \overline{g(3,3)} \circ \overline{g(4,4)} \ldots$
		\end{center}
	
		Como é possível construir $x$, concluímos que a suposta bijeção $f$ não existe. Logo $\mathcal{B}$ é incontável $\blacksquare$
	}
	
	\newpage
	
	\item (5,0 pt) Seja $A = \{\langle R,S \rangle$ | $R$ e $S$ são expressões regulares e $L(R) \subseteq L(S)\}$. Mostre que $A$ é decidível.
	
	\vspace{0.3cm}
	
	{\color{blue}
		R - Para que $L(R) \subseteq L(S)$, é necessário garantir que $L(R) \cup L(S) = L(S)$ (pois toda cadeia em $L(R)$ tem que pertencer também a $L(S)$). Para isto, criamos os AFDs $T$ e $U$ de forma que $L(T) = L(R)$ e $L(U) = L(S)$ (Definição 1.16, e Teorema 1.54). Por fim, criamos o AFD $V$ de forma que $L(V) = L(T) \cup L(U)$ (Teorema 1.25) e verificamos se $\langle U, V \rangle$ é membro de $EQ_{AFD}$ (Teorema 4.5).
		
		Diante disto, será construído a seguir um decisor $M_A$ para $A$ :
		
		$M_A$ = ``Sobre a entrada $\langle R, S \rangle$, em que $R$ e $S$ são expressões regulares, faça:
		\begin{enumerate}
			\item Construa os AFDs $U$ e $V$ conforme descritos anteriormente;
			\item Construa a MT $X$ que decide $EQ_{AFD}$ (Teorema 4.5);
			\item Rode $X$ sobre $\langle U, V \rangle$;
			\begin{enumerate}
				\item Se $X$ aceita, {\it aceite};
				\item Caso contrário, {\it rejeite}.
			\end{enumerate}					
		\end{enumerate}
		
		A linguagem $A$ é decidível pois foi possível construir uma máquina de Turing que a decide (Definição 3.6) $\blacksquare$
		
	}
	
\end{enumerate}

\newpage

\section*{Teoremas Auxiliares}

\begin{itemize}
	
	\item[] {\bf Definição 1.16:} Uma linguagem é chamada de uma linguagem regular se algum autômato finito a reconhece.
	\item[] {\bf Teorema 1.25:} A classe de linguagens regulares é fechada sob a operação de união.
	\item[] {\bf Teorema 1.26:} A classe de linguagens regulares é fechada sob a operação de concatenação.
	\item[] {\bf Teorema 1.39:} Todo autômato finito não-determinístico tem um autômato finito determinístico
	equivalente.
	\item[] {\bf Teorema 1.49:} A classe de linguagens regulares é fechada sob a operação estrela.
	\item[] {\bf Teorema 1.54:} Uma linguagem é regular se e somente se alguma expressão regular a descreve.
	\item[] {\bf Definição 3.5:} Chame uma linguagem de Turing-reconhecível se alguma máquina de Turing a reconhece.
	\item[] {\bf Definição 3.6:} Chame uma linguagem de Turing-decidível ou simplesmente decidível se alguma máquina de Turing a decide.
	\item[] {\bf Teorema 3.13:} Toda máquina de Turing multifita tem uma máquina de Turing que lhe é equivalente.
	\item[] {\bf Teorema 3.16:} Toda máquina de Turing não-determinística tem uma máquina de Turing determinística que lhe é equivalente.
	\item[] {\bf Teorema 3.21:} Uma linguagem é Turing-reconhecível se e somente se algum enumerador a enumera.
	\item[] {\bf Teorema 4.1:} $A_{AFD}$ é uma linguagem decidível.
	\item[] {\bf Teorema 4.2:} $A_{AFN}$ é uma linguagem decidível.
	\item[] {\bf Teorema 4.3:} $A_{EXR}$ é uma linguagem decidível.
	\item[] {\bf Teorema 4.4:} $V_{AFD}$ é uma linguagem decidível.
	\item[] {\bf Teorema 4.5:} $EQ_{AFD}$ é uma linguagem decidível.
	\item[] {\bf Teorema 4.9:} Toda linguagem livre-de-contexto é decidível.
	\item[] {\bf Teorema 4.11:} $A_{MT}$ é uma linguagem indecidível.
	\item[] {\bf Definição 4.14:} Um conjunto $A$ é contável se é finito ou tem o mesmo tamanho que $N$.
	\item[] {\bf Teorema 4.15:} $\mathbb{Q}$ é contável.
	\item[] {\bf Teorema 4.17:} $\mathbb{R}$ é incontável.
	\item[] {\bf Corolário 4.18:} Algumas linguagens não são Turing-reconhecíveis.
	\item[] {\bf Teorema 4.22:} Uma linguagem é decidível sse ela é Turing-reconhecível e co-Turing-reconhecível.
	\item[] {\bf Corolário 4.23:} $\overline{A_{MT}}$ não é Turing-reconhecível.
\end{itemize}


\end{document}