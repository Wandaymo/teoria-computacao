\documentclass[12pt,a4paper,oneside]{article}

\usepackage[utf8]{inputenc}
\usepackage[portuguese]{babel}
\usepackage[T1]{fontenc}
\usepackage{amsmath}
\usepackage{amsfonts}
\usepackage{amssymb}
\usepackage{graphicx}
\usepackage{xcolor}
% Definindo novas cores
\definecolor{verde}{rgb}{0.25,0.5,0.35}

\author{\\Universidade Federal de Goiás (UFG) - Regional  Jataí\\Bacharelado em Ciência da Computação \\Teoria da Computação \\Esdras Lins Bispo Jr.}

\date{21 de agosto de 2017}

\title{\sc \huge Quarto Teste}

\begin{document}

\maketitle

{\bf ORIENTAÇÕES PARA A RESOLUÇÃO}

\small
 
\begin{itemize}
	\item A avaliação é individual, sem consulta;
	\item A pontuação máxima desta avaliação é 10,0 (dez) pontos, sendo uma das 06 (seis) componentes que formarão a média final da disciplina: quatro testes, uma prova e exercícios;
	\item A média final ($MF$) será calculada assim como se segue
	\begin{eqnarray}
		MF & = & MIN(10, S) \nonumber \\
		S & = & (\sum_{i=1}^{4} 0,2.T_i ) + 0,2.P  + 0,1.EA + EB\nonumber
	\end{eqnarray}
	em que 
	\begin{itemize}
		\item $S$ é o somatório da pontuação de todas as avaliações,
		\item $T_i$ é a pontuação obtida no teste $i$,
		\item $P$ é a pontuação obtida na prova,
		\item $EA$ é a pontuação total dos exercícios de aquecimentos, e
		\item $EB$ é a pontuação total dos exercícios-bônus.
	\end{itemize}
	\item O conteúdo exigido desta avaliação compreende o seguinte ponto apresentado no Plano de Ensino da disciplina:  (3) Problemas Decidíveis, (4) Problemas Indecidíveis e (5) Complexidade de Tempo.
\end{itemize}

\begin{center}
	\fbox{\large Nome: \hspace{10cm}}
\end{center}\newpage

\begin{enumerate}
	
	\section*{Quarto Teste}
	
	\item (5,0 pt)  Seja $TODAS_{AFD} = \{ \langle A \rangle$ | $A$ é um AFD e $L(A) = \Sigma^*\}$. Mostre que $TODAS_{AFD} \in$ {\bf P}.
	
	\item (5,0 pt) Mostre que {\bf NP} é fechada sob operação de união.
	
\end{enumerate}

\section*{Teoremas Auxiliares}

\begin{itemize}
	
	\item[] {\bf Definição 1.16:} Uma linguagem é chamada de uma linguagem regular se algum autômato finito a reconhece.
	\item[] {\bf Teorema 1.25:} A classe de linguagens regulares é fechada sob a operação de união.
	\item[] {\bf Teorema 1.26:} A classe de linguagens regulares é fechada sob a operação de concatenação.
	\item[] {\bf Teorema 1.39:} Todo autômato finito não-determinístico tem um autômato finito determinístico
	equivalente.
	\item[] {\bf Teorema 1.49:} A classe de linguagens regulares é fechada sob a operação estrela.
	\item[] {\bf Teorema 1.54:} Uma linguagem é regular se e somente se alguma expressão regular a descreve.
	\item[] {\bf Definição 3.5:} Chame uma linguagem de Turing-reconhecível se alguma máquina de Turing a reconhece.
	\item[] {\bf Definição 3.6:} Chame uma linguagem de Turing-decidível ou simplesmente decidível se alguma máquina de Turing a decide.
	\item[] {\bf Teorema 3.13:} Toda máquina de Turing multifita tem uma máquina de Turing que lhe é equivalente.
	\item[] {\bf Teorema 3.16:} Toda máquina de Turing não-determinística tem uma máquina de Turing determinística que lhe é equivalente.
	\item[] {\bf Teorema 3.21:} Uma linguagem é Turing-reconhecível se e somente se algum enumerador a enumera.
	\item[] {\bf Teorema 4.1:} $A_{AFD}$ é uma linguagem decidível.
	\item[] {\bf Teorema 4.2:} $A_{AFN}$ é uma linguagem decidível.
	\item[] {\bf Teorema 4.3:} $A_{EXR}$ é uma linguagem decidível.
	\item[] {\bf Teorema 4.4:} $V_{AFD}$ é uma linguagem decidível.
	\item[] {\bf Teorema 4.5:} $EQ_{AFD}$ é uma linguagem decidível.
	\item[] {\bf Teorema 4.9:} Toda linguagem livre-de-contexto é decidível.
	\item[] {\bf Teorema 4.11:} $A_{MT}$ é uma linguagem indecidível.
	\item[] {\bf Definição 4.14:} Um conjunto $A$ é contável se é finito ou tem o mesmo tamanho que $N$.
	\item[] {\bf Teorema 4.15:} $\mathbb{Q}$ é contável.
	\item[] {\bf Teorema 4.17:} $\mathbb{R}$ é incontável.
	\item[] {\bf Corolário 4.18:} Algumas linguagens não são Turing-reconhecíveis.
	\item[] {\bf Teorema 4.22:} Uma linguagem é decidível sse ela é Turing-reconhecível e co-Turing-reconhecível.
	\item[] {\bf Corolário 4.23:} $\overline{A_{MT}}$ não é Turing-reconhecível.
	\item[] {\bf Teorema 7.8: }
		Seja $t(n)$ uma função, em que $t(n) \geq n$. Então toda máquina de Turing multifita de tempo $t(n)$ tem uma máquina de Turing de um única fita equivalente de tempo $O(t^2(n))$.
	\item[] {\bf Teorema 7.11: }
		Seja $t(n)$ uma função, em que $t(n) \geq n$. Então toda máquina de Turing não-determinística de uma única fita de tempo $t(n)$ tem uma máquina de Turing de um única fita equivalente de tempo $2^{O(t(n))}$.	
	\item[] {\bf Definição 7.12: }
			{\bf P} é a classe de linguagens que são decidíveis em tempo polinomial sobre uma máquina de Turing determinística de uma única fita. Em outras palavras, {\bf P} = $\bigcup\limits_{k}$ {\bf TIME ($n^k$)}.
	\item[] {\bf Definição 7.19: }
			{\bf NP} é a classe das linguagens que têm verificadores de tempo polinomial.
	\item[] {\bf Teorema 7.20: }
			Uma linguagem está em {\bf NP} sse ela é decidida por alguma máquina de Turing não-determinística de tempo polinomial.  Em outras palavras, {\bf NP} = $\bigcup\limits_{k}$ {\bf NTIME ($n^k$)}.
\end{itemize}


\end{document}