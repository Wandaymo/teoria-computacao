\documentclass[12pt,a4paper,oneside]{article}

\usepackage[utf8]{inputenc}
\usepackage[portuguese]{babel}
\usepackage[T1]{fontenc}
\usepackage{amsmath}
\usepackage{amsfonts}
\usepackage{amssymb}

\usepackage{multirow}
\usepackage{array,graphicx}

\usepackage{xcolor}
% Definindo novas cores
\definecolor{verde}{rgb}{0.25,0.5,0.35}
\definecolor{jpurple}{rgb}{0.5,0,0.35}

\author{\\Universidade Federal de Goiás (UFG) - Regional Jataí\\Bacharelado em Ciência da Computação \\Teoria da Computação - 2017.1 \\Prof. Esdras Lins Bispo Jr.}
\date{}

\title{
	\sc \huge Listas de Exercícios
	\\{\tt Versão 4.0}
}

\begin{document}

\maketitle

\section{Livro de Referência}
	\begin{itemize}
		\item SIPSER, M. {\bf Introdução à Teoria da Computação}, 2a Edição, Editora Thomson Learning, 2011. \color{blue}{\bf Código Bib.: [004 SIP/int]}.
	\end{itemize}
	
\section{Listas de Exercícios}

\begin{enumerate}

	\subsection{Teste 1}
	\item[] {\bf Lista de Exercícios 01:} 3.1, 3.2 (a, c, e), 3.9, 3.15;
	
	\subsection{Teste 2}
	
	\item[] {\bf Lista de Exercícios 02:} 3.6, 3.7, 3.8, 3.16;
	\item[] {\bf Lista de Exercícios 03:} 4.1, 4.2, 4.3, 4.9;
	
	\subsection{Teste 3}
		
	\item[] {\bf Lista de Exercícios 04:} 4.6, 4.7, 4.11, 4.12, 4.18;
	%\item[] {\bf Lista de Exercícios 05:} 7.1, 7.2, 7.6;
	
\end{enumerate}

\end{document}