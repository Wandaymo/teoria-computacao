\documentclass[12pt,a4paper,oneside]{article}

\usepackage[utf8]{inputenc}
\usepackage[portuguese]{babel}
\usepackage[T1]{fontenc}
\usepackage{amsmath}
\usepackage{amsfonts}
\usepackage{amssymb}
\usepackage{graphicx}
\usepackage{xcolor}
% Definindo novas cores
\definecolor{verde}{rgb}{0.25,0.5,0.35}

\author{\\Universidade Federal de Goiás (UFG) - Regional Jataí Jataí\\Bacharelado em Ciência da Computação \\Teoria da Computação \\Esdras Lins Bispo Jr.}

\date{12 de setembro de 2016}

\title{\sc \huge Prova (Extra)}

\begin{document}

\maketitle

{\bf ORIENTAÇÕES PARA A RESOLUÇÃO}
{ \footnotesize
\begin{itemize}
	\item A avaliação é individual, sem consulta;
	\item A pontuação máxima desta avaliação é 10,0 (dez) pontos.
	\item A média final ($MF$) será calculada assim como se segue
	\begin{eqnarray}
		MF & = & \left\{
					\begin{array}{ll}
						6,0  & \mbox{, se } PE \geq NM \\
						MA & \mbox{, se } PE < NM
					\end{array}
				\right. \nonumber \\
		NM & = & 10 - MA \nonumber
	\end{eqnarray}
	em que 
	\begin{itemize}
		\item $PE$ é a pontuação obtida na prova extra,
		\item $MF$ é a média final na disciplina,
		\item $MA$ é a média atual na disciplina, e
		\item $NM$ é a pontuação mínima a ser obtida na prova extra.
	\end{itemize}
	\item O conteúdo exigido desta avaliação compreende todos os pontos apresentados no Plano de Ensino da disciplina.
\end{itemize}

\begin{center}
	\fbox{\large Nome: \hspace{10cm}}
	\fbox{\large Assinatura: \hspace{9cm}}
\end{center}
}
\newpage

\begin{enumerate}
	
	\item (2,5 pt) É verdade que se uma linguagem é Turing-reconhecível então algum enumerador a enumera. Por que não podemos usar o algoritmo do enumerador abaixo para provar esta afirmação? 
	
	Seja $s_1, s_2, \ldots$ uma lista de todas as cadeias em $\Sigma^*$. Seja $M$ uma máquina de Turing que reconhece uma dada linguagem.

E = ``Ignore a entrada.
	\begin{enumerate}
		\item Repita o que se segue para $i = 1,2,3, \ldots$;
		\item Rode $M$ sobre $s_i$;
		\item Se ela aceita, imprima $s_i$''.
	\end{enumerate}
	
	\vspace{0.3cm}
	
	{ \color{blue}
	R - Por que a máquina de Turing $M$ pode, ao executar o passo (b), entrar em {\it loop}. Se isto ocorrer em um dado momento, pode ser que haja alguma cadeia, que pertença a $L(M)$, que ainda não foi iterada. Logo, esta cadeia não seria impressa e este enumerador não cumpriria com o prometido.
	}
	
	\item (2,5 pt)  Seja $A = \{ \langle B \rangle$ | $B$ é um AFD e $L(B) = 1^* \}$. Mostre que $A$ é decidível.
	
	\vspace{0.3cm}
	
	{\color{blue}
		R - Primeiro, será criado um AFD $S$ de forma que $L(S) = 1^*$. Isto é possível, pois $1^*$ é regular (Definição 1.16). Segundo, será construído um decisor $M_A$ para $A$, conforme descrito a seguir:
		
		$M_A$ = ``Sobre a entrada $\langle B \rangle$, em que $B$ é um AFD, faça:
			\begin{enumerate}
				\item Construa o AFD $S$ conforme descrito anteriormente;
				\item Construa a MT $T$ que decide $EQ_{AFD}$ (Teorema 4.5);
				\item Rode $T$ sobre $\langle B, S \rangle$;
				\begin{enumerate}
					\item Se $T$ aceita, {\it aceite};
					\item Caso contrário, {\it rejeite}.
				\end{enumerate}					
			\end{enumerate}
		
		A linguagem $A$ é decidível, pois foi possível construir uma máquina de Turing que a decide (Definição 3.6) $\blacksquare$
		
	}	
	
	\newpage
	
	\item (2,5 pt)  Seja $A = \{ \langle B,C \rangle$ | $B$ é um AFN, $C$ é uma expressão regular e $L(B) \cap L(C) \not= \emptyset \}$. Mostre que $A$ é decidível.
	
	\vspace{0.3cm}
	
	\vspace{0.3cm}
	
	{\color{blue}
		R - Primeiro, será criado um AFD $S$ de forma que $L(S) = L(B) \cap L(C)$. Isto é possível pois $L(B)$ e $L(C)$ são regulares e a classe de linguagens regulares é fechada sob a operação de intersecção (Teorema 1.39, Teorema 1.54, Definição 1.16 e Teorema 1.49.1). Segundo, será construído um decisor $M_A$ para $A$, como se segue:
		
		$M_A$ = ``Sobre a entrada $\langle B, C \rangle$, em que $B$ é um AFN e $C$ é uma expressão regular, faça:
			\begin{enumerate}
				\item Construa o AFD $S$ conforme descrito anteriormente;
				\item Construa a MT $T$ que decide $V_{AFD}$ (Teorema 4.4);
				\item Rode $T$ sobre $\langle S \rangle$;
				\begin{enumerate}
					\item Se $T$ aceita, {\it rejeite};
					\item Caso contrário, {\it aceita}.
				\end{enumerate}					
			\end{enumerate}
		
		A linguagem $A$ é decidível, pois foi possível construir uma máquina de Turing que a decide (Definição 3.6) $\blacksquare$
		
	}
	
	\newpage	
	
	\item (2,5 pt) Mostre que {\bf NP} é fechada sob operação de união.
	
	\vspace{0.3cm}
	
	{\color{blue}
		{\bf Prova:} Sejam $A$ e $B$ duas linguagens decidíveis em $NP$. Sejam $M_A$ e $M_B$ duas máquinas de Turing não-determinísticas que decidem as linguagens $A$ e $B$, respectivamente (pois se uma linguagem é decidível, então uma máquina de Turing a decide). Como $A$ e $B$ são decidíveis em tempo polinomial não-determinístico, $A$ e $B$ pertencem a {\sc NTIME}$(n^k)$ e {\sc NTIME}$(n^l)$ respectivamente (em que $k$ e $l$ são números naturais).  Iremos construir a máquina de Turing não-determinística $M_{aux}$, a partir de $M_A$ e $M_B$, que decide $A \cup B$ em tempo polinomial não-determi\-nís\-ti\-co. A descrição de $M_{aux}$ é dada a seguir:
			
			$M_{aux}$ = ``Sobre a entrada $\omega$, faça:
			\begin{enumerate}
				\item Rode $M_A$ sobre $\omega$.
				\item Rode $M_B$ sobre $\omega$.
				\item Se $M_A$ ou $M_B$ aceita, {\it aceite}.
				\item Caso contrário, {\it rejeite}''.
			\end{enumerate}
			
			O tempo de execução $t$ de $M_{aux}$ é igual a soma do tempo de execução dos passos (a), (b), (c) e (d). Logo, $t = O(n^k) + O(n^l) + O(1) + O(1) = O(n^{max(k,l)})$. 
			
			Seja  $c = max(k,l)$. Temos assim, $t = O(n^c)$. Como $c$ é um número natural, $A \cup B \in$ {\sc NTIME}$(n^c)$ e, consequentemente, $A \cup B \in NP$. Logo, podemos afirmar que $NP$ é fechada sob a operação de união $\blacksquare$
	}
	
\end{enumerate}

\section*{Teoremas Auxiliares}

\begin{itemize}
	
	\item[] {\bf Definição 1.16:} Uma linguagem é chamada de uma linguagem regular se algum autômato finito a reconhece.
	\item[] {\bf Teorema 1.25:} A classe de linguagens regulares é fechada sob a operação de união.
	\item[] {\bf Teorema 1.26:} A classe de linguagens regulares é fechada sob a operação de concatenação.
	\item[] {\bf Teorema 1.26.1:} A classe de linguagens regulares é fechada sob a operação de complemento.
	\item[] {\bf Teorema 1.39:} Todo autômato finito não-determinístico tem um autômato finito determinístico
	equivalente.
	\item[] {\bf Teorema 1.49:} A classe de linguagens regulares é fechada sob a operação estrela.
	\item[] {\bf Teorema 1.49.1:} A classe de linguagens regulares é fechada sob a operação de intersecção.
	\item[] {\bf Teorema 1.54:} Uma linguagem é regular se e somente se alguma expressão regular a descreve.
	\item[] {\bf Definição 3.5:} Chame uma linguagem de Turing-reconhecível se alguma máquina de Turing a reconhece.
	\item[] {\bf Definição 3.6:} Chame uma linguagem de Turing-decidível ou simplesmente decidível se alguma máquina de Turing a decide.
	\item[] {\bf Teorema 3.13:} Toda máquina de Turing multifita tem uma máquina de Turing que lhe é equivalente.
	\item[] {\bf Teorema 3.16:} Toda máquina de Turing não-determinística tem uma máquina de Turing determinística que lhe é equivalente.
	\item[] {\bf Teorema 3.21:} Uma linguagem é Turing-reconhecível se e somente se algum enumerador a enumera.
	\item[] {\bf Teorema 4.1:} $A_{AFD}$ é uma linguagem decidível.
	\item[] {\bf Teorema 4.2:} $A_{AFN}$ é uma linguagem decidível.
	\item[] {\bf Teorema 4.3:} $A_{EXR}$ é uma linguagem decidível.
	\item[] {\bf Teorema 4.4:} $V_{AFD}$ é uma linguagem decidível.
	\item[] {\bf Teorema 4.5:} $EQ_{AFD}$ é uma linguagem decidível.
	\item[] {\bf Teorema 4.9:} Toda linguagem livre-de-contexto é decidível.
	\item[] {\bf Teorema 4.11:} $A_{MT}$ é uma linguagem indecidível.
	\item[] {\bf Definição 4.14:} Um conjunto $A$ é contável se é finito ou tem o mesmo tamanho que $N$.
	
\end{itemize}

\end{document}