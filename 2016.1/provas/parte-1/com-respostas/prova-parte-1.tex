\documentclass[12pt,a4paper,oneside]{article}

\usepackage[utf8]{inputenc}
\usepackage[portuguese]{babel}
\usepackage[T1]{fontenc}
\usepackage{amsmath}
\usepackage{amsfonts}
\usepackage{amssymb}
\usepackage{graphicx}
\usepackage{xcolor}
% Definindo novas cores
\definecolor{verde}{rgb}{0.25,0.5,0.35}

\author{\\Universidade Federal de Goiás (UFG) - Regional Jataí Jataí\\Bacharelado em Ciência da Computação \\Teoria da Computação \\Esdras Lins Bispo Jr.}

\date{15 de agosto de 2016}

\title{\sc \huge Prova (Parte 1)}

\begin{document}

\maketitle

{\bf ORIENTAÇÕES PARA A RESOLUÇÃO}

\begin{itemize}
	\item A avaliação é individual, sem consulta;
	\item A pontuação máxima desta avaliação é 10,0 (dez) pontos, sendo uma das 06 (seis) componentes que formarão a média final da disciplina: quatro testes, uma prova e exercícios;
	\item A média final ($MF$) será calculada assim como se segue
	\begin{eqnarray}
		MF & = & MIN(10, S) \nonumber \\
		S & = & (\sum_{i=1}^{4} 0,2.T_i ) + 0,2.P  + EB \nonumber
	\end{eqnarray}
	em que 
	\begin{itemize}
		\item $S$ é o somatório da pontuação de todas as avaliações,
		\item $T_i$ é a pontuação obtida no teste $i$,
		\item $P$ é a pontuação obtida na prova, e
		\item $EB$ é a pontuação total dos exercícios-bônus.
	\end{itemize}
	\item O conteúdo exigido desta avaliação compreende o seguinte ponto apresentado no Plano de Ensino da disciplina:  (1) Teoria da Computação, (2) Modelos de Computação e (3) Problemas Decidíveis.
\end{itemize}

\begin{center}
	\fbox{\large Nome: \hspace{10cm}}
	\fbox{\large Assinatura: \hspace{9cm}}
\end{center}

\newpage

\begin{enumerate}
	
	\section*{Primeiro Teste}
	
	\item (5,0 pt) Explique porque a descrição abaixo não é uma descrição de uma máquina de Turing legítima.
	
$M_{ruim}$ = ``A entrada é um polinômio $p$ sobre as variáveis $x_1, \ldots, x_k$ .
	\begin{enumerate}
		\item Tente todas as possíveis valorações de $x_1, \ldots, x_k$ para valores inteiros.
		\item Calcule o valor de $p$ sobre todas essas valorações.
		\item Se alguma dessas valorações torna o valor de $p$ igual a 0, aceite; caso contrário, rejeite.''
	\end{enumerate}
	
	{ \color{verde}
	R - O problema está nos passos (a) e (b). Testar todas as possíveis valorações para $k$ variáveis inteiras pode ser realizada de forma que o passo (c) nunca seja executado. Se admitirmos $k=1$ e escolhêssemos a seguinte estratégia:
		\begin{enumerate}
			\item[(i)] Faça o teste para $x_1 = 0$;
			\item[(ii)] Faça o teste com $x_1$ para todos os inteiros negativos;
			\item[(iii)] Faça o teste com $x_1$ para todos os inteiros positivos.
		\end{enumerate}		 
		o passo (iii) nunca seria executado, pois o passo (ii) entraria em um laço infinito (pois existem infinitos valores inteiros negativos). Logo, a descrição dos passos (a) e (b) estão inadequadas, não fornecendo informações suficientes para, se necessário, construir a máquina de Turing em um nível formal.
	}
	
	\newpage
	
	\item (5,0 pt) A operação binária ou-exclusivo, representada pelo símbolo $\otimes$, é definida da seguinte forma:
		\begin{center}
			$X \otimes Y = (\overline{X} \cap Y) \cup (X \cap \overline{Y})$
		\end{center}
		em que $X$ e $Y$ são dois conjuntos quaisquer.
				 
		Mostre que a classe de linguagens decidíveis é fechada sob a operação de ou-exclusivo.

		\vspace{0.3cm}
		
		{\color{verde}	
				{\bf Prova:} Sejam $A$ e $B$ duas linguagens decidíveis. É possível construir duas máquinas de Turing $M_A$ e $M_B$ que decidem as linguagens $A$ e $B$, respectivamente (Definição 3.6). Iremos construir a máquina de Turing $M_{aux}$, a partir de $M_A$ e $M_B$, que decide $A \otimes B$. A descrição de $M_{aux}$ é dada a seguir:
			
			$M_{aux}$ = ``Sobre a entrada $\omega$, faça:
			\begin{enumerate}
				\item Rode $M_A$ sobre $\omega$; 
				\item Rode $M_B$ sobre $\omega$; 
				\item Se $M_A$ rejeita e $M_B$ aceita, {\it aceite};
				\item Se $M_A$ aceita e $M_B$ rejeita, {\it aceite};
				\item {\it Rejeite}.''.
			\end{enumerate}
			
			Como foi possível construir $M_{aux}$, então $A \otimes B$ é decidível. Ora, se $A \otimes B$ é decidível, então a classe de linguagens decidíveis é fechada sob a operação de ou-exclusivo $\blacksquare$
			}
			
			\newpage
		
	
	\section*{Segundo Teste}
	
	\item (5,0 pt) Mostre que a classe de linguagens Turing-reconhecíveis é fechada sob a operação de interseção.
	
	\vspace{0.3cm}
		
		{\color{verde}	
				{\bf Prova:} Sejam $A$ e $B$ duas linguagens Turing-reconhecíveis. É possível construir duas máquinas de Turing $M_A$ e $M_B$ que reconhecem as linguagens $A$ e $B$, respectivamente (Definição 3.5). Iremos construir a máquina de Turing $M_{aux}$, a partir de $M_A$ e $M_B$, que reconhece $A \cap B$. A descrição de $M_{aux}$ é dada a seguir:
			
			$M_{aux}$ = ``Sobre a entrada $\omega$, faça:
			\begin{enumerate}
				\item Rode $M_A$ sobre $\omega$; 
				\item Rode $M_B$ sobre $\omega$; 
				\item Se $M_A$ e $M_B$ aceitam, {\it aceite};
				\item Caso contrário, {\it rejeite}''.
			\end{enumerate}
			
			Como foi possível construir $M_{aux}$, então $A \cap B$ é Turing-reconhecível. Ora, se $A \cap B$ é Tu\-ring-reconhecível, então a classe de linguagens Tu\-ring-reconhecíveis é fechada sob a operação de intersecção $\blacksquare$
			}
	
	\newpage	
	
	\item (5,0 pt) Considere o problema de se determinar se um AFD e uma expressão regular são equivalentes. Expresse esse problema como uma linguagem e mostre que ele é decidível.
	
	\vspace{0.3cm}	
	
	{ \color{verde}
	R - Este problema pode ser expresso pela linguagem a seguir:
		\begin{center}
			$EQ_{AFD-ER} = \{ \langle A, B\rangle$ | $A$ é um AFD, $B$ é uma expressão regular e $L(A) = L(B) \}$
		\end{center}
		em que qualquer par $\langle A, B\rangle \in EQ_{AFD-ER}$, se o AFD $A$ e a expressão regular $B$ são equivalentes.
		
		Pode-se mostrar que $EQ_{AFD-ER}$ é decidível construindo uma máquina de Turing $M$ que a decida (Definição 3.6). A descrição de $M$ é dada a seguir:
			
			$M$ = ``Sobre a entrada $\langle A, B\rangle$, em que $A$ é um AFD e \\ 
			$B$ é uma expressão regular, faça:
			\begin{enumerate}
				\item Converta a expressão regular $B$ no AFD equivalente $C$ \\(Teorema 1.54 e Definição 1.16); 
				\item Construa a MT $T$ que decide $EQ_{AFD}$ (Teorema 4.5); 
				\item  Rode $T$ sobre $\langle A, C\rangle$:
					\begin{enumerate}
						\item Se $T$ aceita, {\it aceite};
						\item Caso contrário, {\it rejeite}.''
					\end{enumerate}
			\end{enumerate}
			
			Como foi possível construir $M$, então $EQ_{AFD-ER}$ é decidível $\blacksquare$
	}
	
\end{enumerate}

\section*{Teoremas Auxiliares}

\begin{itemize}
	
	\item[] {\bf Definição 1.16:} Uma linguagem é chamada de uma linguagem regular se algum autômato finito a reconhece.
	\item[] {\bf Teorema 1.25:} A classe de linguagens regulares é fechada sob a operação de união.
	\item[] {\bf Teorema 1.26:} A classe de linguagens regulares é fechada sob a operação de concatenação.
	\item[] {\bf Teorema 1.26.1:} A classe de linguagens regulares é fechada sob a operação de complemento.
	\item[] {\bf Teorema 1.39:} Todo autômato finito não-determinístico tem um au\-tômato finito determinístico
	equivalente.
	\item[] {\bf Teorema 1.49:} A classe de linguagens regulares é fechada sob a operação estrela.
	\item[] {\bf Teorema 1.49.1:} A classe de linguagens regulares é fechada sob a operação de intersecção.
	\item[] {\bf Teorema 1.54:} Uma linguagem é regular se e somente se alguma expressão regular a descreve.
	\item[] {\bf Definição 3.5:} Chame uma linguagem de Turing-reconhecível se alguma máquina de Turing a reconhece.
	\item[] {\bf Definição 3.6:} Chame uma linguagem de Turing-decidível ou simplesmente decidível se alguma máquina de Turing a decide.
	\item[] {\bf Teorema 3.13:} Toda máquina de Turing multifita tem uma máquina de Turing que lhe é equivalente.
	\item[] {\bf Teorema 3.16:} Toda máquina de Turing não-determinística tem uma máquina de Turing determinística que lhe é equivalente.
	\item[] {\bf Teorema 3.21:} Uma linguagem é Turing-reconhecível se e somente se algum enumerador a enumera.
	\item[] {\bf Teorema 4.1:} $A_{AFD}$ é uma linguagem decidível.
	\item[] {\bf Teorema 4.2:} $A_{AFN}$ é uma linguagem decidível.
	\item[] {\bf Teorema 4.3:} $A_{EXR}$ é uma linguagem decidível.
	\item[] {\bf Teorema 4.4:} $V_{AFD}$ é uma linguagem decidível.
	\item[] {\bf Teorema 4.5:} $EQ_{AFD}$ é uma linguagem decidível.
	\item[] {\bf Teorema 4.9:} Toda linguagem livre-de-contexto é decidível.
	\item[] {\bf Teorema 4.11:} $A_{MT}$ é uma linguagem indecidível.
	\item[] {\bf Definição 4.14:} Um conjunto $A$ é contável se é finito ou tem o mesmo tamanho que $N$.
	
\end{itemize}

\end{document}