\documentclass[12pt,a4paper,oneside]{article}

\usepackage[utf8]{inputenc}
\usepackage[portuguese]{babel}
\usepackage[T1]{fontenc}
\usepackage{amsmath}
\usepackage{amsfonts}
\usepackage{amssymb}
\usepackage{graphicx}
\usepackage{xcolor}
% Definindo novas cores
\definecolor{verde}{rgb}{0.25,0.5,0.35}

\author{\\Universidade Federal de Goiás (UFG) - Regional Jataí \\Bacharelado em Ciência da Computação \\Teoria da Computação \\Esdras Lins Bispo Jr.}

\date{02 de agosto de 2016}

\title{\sc \huge Quarto Teste}

\begin{document}

\maketitle

{\bf ORIENTAÇÕES PARA A RESOLUÇÃO}

\begin{itemize}
	\item A avaliação é individual, sem consulta;
	\item A pontuação máxima desta avaliação é 10,0 (dez) pontos, sendo uma das 06 (seis) componentes que formarão a média final da disciplina: quatro testes, uma prova e exercícios;
	\item A média final ($MF$) será calculada assim como se segue
	\begin{eqnarray}
		MF & = & MIN(10, S) \nonumber \\
		S & = & (\sum_{i=1}^{4} 0,2.T_i ) + 0,2.P  + 0,1.E \nonumber
	\end{eqnarray}
	em que 
	\begin{itemize}
		\item $S$ é o somatório da pontuação de todas as avaliações,
		\item $T_i$ é a pontuação obtida no teste $i$,
		\item $P$ é a pontuação obtida na prova, e
		\item $E$ é a pontuação total dos exercícios.
	\end{itemize}
	\item O conteúdo exigido desta avaliação compreende o seguinte ponto apresentado no Plano de Ensino da disciplina: (5) Complexidade de Tempo e (6) NP-Completude.
\end{itemize}

\begin{center}
	\fbox{\large Nome: \hspace{10cm}}
	\fbox{\large Assinatura: \hspace{9cm}}
\end{center}

\newpage

\begin{enumerate}
	
	\section*{Quarto Teste}
	
	\item (5,0 pt) Mostre que {\bf NP} é fechada sob a operação de concatenação.
	
	\item (5,0 pt) Um triângulo em um grafo não-direcionado é um 3-clique. Mostre que TRIANGULO $\in$ {\bf P}, em que 
	\begin{center}
		TRIANGULO = $\{ \langle G \rangle$ \mbox{ | } $G$ contém um triângulo $\}$.
	\end{center}
	
\end{enumerate}

\end{document}