\documentclass[12pt,a4paper,oneside]{article}

\usepackage[utf8]{inputenc}
\usepackage[portuguese]{babel}
\usepackage[T1]{fontenc}
\usepackage{amsmath}
\usepackage{amsfonts}
\usepackage{amssymb}
\usepackage{graphicx}
\usepackage{xcolor}
% Definindo novas cores
\definecolor{verde}{rgb}{0.25,0.5,0.35}

\author{\\Universidade Federal de Goiás (UFG) - Regional Jataí \\Bacharelado em Ciência da Computação \\Teoria da Computação \\Esdras Lins Bispo Jr.}

\date{02 de agosto de 2016}

\title{\sc \huge Quarto Teste}

\begin{document}

\maketitle

{\bf ORIENTAÇÕES PARA A RESOLUÇÃO}

\begin{itemize}
	\item A avaliação é individual, sem consulta;
	\item A pontuação máxima desta avaliação é 10,0 (dez) pontos, sendo uma das 06 (seis) componentes que formarão a média final da disciplina: quatro testes, uma prova e exercícios;
	\item A média final ($MF$) será calculada assim como se segue
	\begin{eqnarray}
		MF & = & MIN(10, S) \nonumber \\
		S & = & (\sum_{i=1}^{4} 0,2.T_i ) + 0,2.P  + 0,1.E \nonumber
	\end{eqnarray}
	em que 
	\begin{itemize}
		\item $S$ é o somatório da pontuação de todas as avaliações,
		\item $T_i$ é a pontuação obtida no teste $i$,
		\item $P$ é a pontuação obtida na prova, e
		\item $E$ é a pontuação total dos exercícios.
	\end{itemize}
	\item O conteúdo exigido desta avaliação compreende o seguinte ponto apresentado no Plano de Ensino da disciplina: (5) Complexidade de Tempo e (6) NP-Completude.
\end{itemize}

\begin{center}
	\fbox{\large Nome: \hspace{10cm}}
	\fbox{\large Assinatura: \hspace{9cm}}
\end{center}

\newpage

\begin{enumerate}
	
	\section*{Quarto Teste}
	
	\item (5,0 pt) Mostre que {\bf NP} é fechada sob a operação de concatenação. \\
	
	{\color{verde}
		{\bf Prova:} Sejam $A$ e $B$ duas linguagens decidíveis em $NP$. Sejam $M_A$ e $M_B$ duas máquinas de Turing não-determinísticas que decidem as linguagens $A$ e $B$, respectivamente (pois se uma linguagem é decidível, então uma máquina de Turing a decide). Como $A$ e $B$ são decidíveis em tempo polinomial não-determinístico, $A$ e $B$ pertencem a {\sc NTIME}$(n^k)$ e {\sc NTIME}$(n^l)$ respectivamente (em que $k$ e $l$ são números naturais).  Iremos construir a máquina de Turing não-determinística $M_{aux}$, a partir de $M_A$ e $M_B$, que decide $A \circ B$ em tempo polinomial não-determinístico. A descrição de $M_{aux}$ é dada a seguir:
			
			$M_{aux}$ = ``Sobre a entrada $\omega$, faça:
			\begin{enumerate}
				\item Não-deterministicamente selecione um corte de $\omega$, \\de forma que $\omega = \omega_A \circ \omega_B$:
				\begin{enumerate}
					\item Rode $M_A$ sobre $\omega_A$.
					\item Rode $M_B$ sobre $\omega_B$.
					\item Se $M_A$ e $M_B$ aceitam, {\it aceite}.
					\item Caso contrário, {\it rejeite}''.
				\end{enumerate}
			\end{enumerate}
			
			O tempo de execução $t$ de $M_{aux}$ é igual a soma do tempo de execução dos passos (a), (i), (ii), (iii) e (iv). Logo, $t = O(n) + O(n^k) + O(n^l) + O(1) + O(1) = O(n^{max(k,l)})$. 
			
			Seja  $c = max(k,l)$. Temos assim, $t = O(n^c)$. Como $c$ é um número natural, $A \circ B \in$ {\sc NTIME}$(n^c)$ e, consequentemente, $A \circ B \in NP$. Logo, podemos afirmar que $NP$ é fechada sob a operação de concatenação $\blacksquare$
	}
	
	\newpage
	
	\item (5,0 pt) Um triângulo em um grafo não-direcionado é um 3-clique. Mostre que TRIANGULO $\in$ {\bf P}, em que 
	\begin{center}
		TRIANGULO = $\{ \langle G \rangle$ \mbox{ | } $G$ contém um triângulo $\}$.
	\end{center}
		{ \color{verde}
				{\bf Prova:} Se TRIANGULO $\in$ {\bf P}, então é possível construir uma máquina de Turing simples que a decide em tempo polinomial. Construiremos $M$ que decide TRIANGULO:
			
			$M$ = ``Sobre a entrada $\langle G \rangle$, em que $G$ é um grafo não-direcionado, faça:
			\begin{enumerate}
				\item Para cada conjunto distinto $C$ com três vértices de $G$, faça:
					\begin{enumerate}
						\item Verifique se $C$ forma um 3-clique em $G$.
						\item Se sim, {\it aceite}.
					\end{enumerate}
				\item {\it Rejeite}''.
			\end{enumerate}
			
			O tempo de execução $t$ de $M$ é igual a soma do tempo de execução dos passos (a) e (b). Logo, $t = O(n^3)(O(n^2) + O(1)) + O(1)= O(n^5)$. 
			
			5 é um número natural e TRIANGULO $\in$ {\sc TIME}$(n^5)$. Logo, podemos afirmar que TRIANGULO $\in$ {\bf P} $\blacksquare$
			}
			
\end{enumerate}

\end{document}