\documentclass[12pt,a4paper,oneside]{article}

\usepackage[utf8]{inputenc}
\usepackage[portuguese]{babel}
\usepackage[T1]{fontenc}
\usepackage{amsmath}
\usepackage{amsfonts}
\usepackage{amssymb}
\usepackage{graphicx}
\usepackage{xcolor}
% Definindo novas cores
\definecolor{verde}{rgb}{0.25,0.5,0.35}

\author{\\Universidade Federal de Goiás (UFG) - Regional  Jataí\\Bacharelado em Ciência da Computação \\Teoria da Computação \\Esdras Lins Bispo Jr.}

\date{07 de março de 2018}

\title{\sc \huge Prova (Parte 2)}

\begin{document}

\maketitle

{\bf ORIENTAÇÕES PARA A RESOLUÇÃO}

\small
 
\begin{itemize}
	\item A avaliação é individual, sem consulta;
	\item A pontuação máxima desta avaliação é 10,0 (dez) pontos, sendo uma das 06 (seis) componentes que formarão a média final da disciplina: quatro testes, uma prova e exercícios;
	\item A média final ($MF$) será calculada assim como se segue
	\begin{eqnarray}
		MF & = & MIN(10, S) \nonumber \\
		S & = & (\sum_{i=1}^{4} 0,2.T_i ) + 0,2.P  + EB\nonumber
	\end{eqnarray}
	em que 
	\begin{itemize}
		\item $S$ é o somatório da pontuação de todas as avaliações,
		\item $T_i$ é a pontuação obtida no teste $i$,
		\item $P$ é a pontuação obtida na prova, e
		\item $EB$ é a pontuação total dos exercícios-bônus.
	\end{itemize}
	\item O conteúdo exigido desta avaliação compreende o seguinte ponto apresentado no Plano de Ensino da disciplina: (3) Problemas Decidíveis, (4) Problemas indecidíveis e (5) Complexidade de Tempo.
\end{itemize}

\begin{center}
	\fbox{\large Nome: \hspace{10cm}}
\end{center}

\newpage

\begin{enumerate}
	
	\section{Terceiro Teste}
	
	\item (5,0 pt)	{\bf [Sipser 4.12]} Seja $A = \{\langle R,S \rangle$ | $R$ e $S$ são expressões regulares e $L(R) \subseteq L(S)\}$. Mostre que $A$ é decidível.
	
	\vspace{0.3cm}
	
	{\color{blue}
		R - Para que $L(R) \subseteq L(S)$, é necessário garantir que $L(R) \cup L(S) = L(S)$ (pois toda cadeia em $L(R)$ tem que pertencer também a $L(S)$). Para isto, criamos os AFDs $T$ e $U$ de forma que $L(T) = L(R)$ e $L(U) = L(S)$ (Definição 1.16, e Teorema 1.54). Por fim, criamos o AFD $V$ de forma que $L(V) = L(T) \cup L(U)$ (Teorema 1.25) e verificamos se $\langle U, V \rangle$ é membro de $EQ_{AFD}$ (Teorema 4.5).
		
		Diante disto, será construído a seguir um decisor $M_A$ para $A$ :
		
		$M_A$ = ``Sobre a entrada $\langle R, S \rangle$, em que $R$ e $S$ são expressões regulares, faça:
		\begin{enumerate}
			\item Construa os AFDs $U$ e $V$ conforme descritos anteriormente;
			\item Construa a MT $X$ que decide $EQ_{AFD}$ (Teorema 4.5);
			\item Rode $X$ sobre $\langle U, V \rangle$;
			\begin{enumerate}
				\item Se $X$ aceita, {\it aceite};
				\item Caso contrário, {\it rejeite}.
			\end{enumerate}					
		\end{enumerate}
		
		A linguagem $A$ é decidível pois foi possível construir uma máquina de Turing que a decide (Definição 3.6) $\blacksquare$
		
	}

	\newpage

	\item (5,0 pt) {\bf [Sipser 4.15]}  Seja $A = \{\langle R \rangle$ | $R$ é uma expressão regular que descreve uma linguagem contendo pelo menos uma cadeia $\omega$ que tem 111 como uma subcadeia (i.e., $\omega = x111y$ para alguma $x$ alguma e $y$)$\}$. Mostre que $A$ é decidível.
	
	\vspace{0.3cm}
	
	{\color{blue} {\bf Resposta:} É possível criar um AFD $B$ de forma que $L(B)$ seja a expressão regular $\Sigma^*111\Sigma^*$ (Definição 1.16 e Teorema 1.54). Assim, para que $R$ gere pelo menos uma cadeia $\omega$ que tem 111 como uma subcadeia, é necessário garantir que $L(R) \cap L(B) \not= \emptyset$. Por fim, vamos criar um outro AFD $C$ de forma que $L(C) = L(R) \cap L(B)$ (pois a classe de linguagens regulares é fechada sob a operação de interseção) e verificamos se $\langle C \rangle$ é membro de $V_{AFD}$ (Teorema 4.4).
		
		Diante disto, será construído a seguir um decisor $M_A$ para $A$ :
		
		$M_A$ = ``Sobre a entrada $\langle R \rangle$, em que $R$ é uma expressão regular, faça:
		\begin{enumerate}
			\item Construa o AFD $B$ conforme descrito anteriormente;
			\item Construa o AFD $C$ conforme descrito anteriormente;
			\item Construa a MT $X$ que decide $V_{AFD}$ (Teorema 4.4);
			\item Rode $X$ sobre $\langle C \rangle$;
			\begin{enumerate}
				\item Se $X$ aceita, {\it rejeite};
				\item Caso contrário, {\it aceite}.
			\end{enumerate}					
		\end{enumerate}
		
		A linguagem $A$ é decidível pois foi possível construir uma máquina de Turing que a decide (Definição 3.6) $\blacksquare$
		
	}

\newpage
	
	\section*{Quarto Teste}
	
	\item (5,0 pt) {\bf [Sipser 7.6]}  Mostre que {\bf P} é fechada sob operação de concatenação.
	
	\vspace*{0.3cm}
	
	{\color{blue}
		{\bf Prova:} Sejam $A$ e $B$ duas linguagens decidíveis em {\bf P}. Sejam $M_A$ e $M_B$ duas máquinas de Turing simples que decidem as linguagens $A$ e $B$, respectivamente (pois se uma linguagem é decidível, então uma máquina de Turing a decide). Como $A$ e $B$ são decidíveis em tempo polinomial determinístico, $A$ e $B$ pertencem a {\sc TIME}$(n^k)$ e {\sc TIME}$(n^l)$ respectivamente (em que $k$ e $l$ são números naturais).  Iremos construir a máquina de Turing $M_{aux}$, a partir de $M_A$ e $M_B$, que decide $A \circ B$ em tempo polinomial. A descrição de $M_{aux}$ é dada a seguir:
		
		$M_{aux}$ = ``Sobre a entrada $\omega$, faça:
		\begin{enumerate}
			\item Para cada um dos $n+1$ cortes de $\omega$, de forma que $\omega = \omega_A \circ \omega_B$, faça:
			\begin{enumerate}
				\item Rode $M_A$ sobre $\omega_A$;
				\item Rode $M_B$ sobre $\omega_B$;
				\item Se $M_A$ e $M_B$ aceitam, {\it aceite}.
			\end{enumerate}
			\item {\it Rejeite}''.
		\end{enumerate}
		
		O tempo de execução $t$ de $M_{aux}$ é igual a soma do tempo de execução dos passos (a) e (b). Logo, $t = O(n) \times (O(n^k) + O(n^l) + O(1)) + O(1) = O(n) \times O(n^{max(k,l)}) = O(n^{max(k,l) + 1})$. 
		
		Seja  $c = max(k,l) + 1$. Temos assim, $t = O(n^c)$. Como $c$ é um número natural, $A \circ B \in$ {\sc TIME}$(n^c)$ e, consequentemente, $A \circ B \in$ {\bf P}. Logo, podemos afirmar que {\bf P} é fechada sob a operação de concatenação $\blacksquare$
	}

\newpage

	\item (5,0 pt)  {\bf [Sipser 7.9] (Adaptação)} Um triângulo em um grafo não-direcionado é um 3-clique. Mostre que $TRIANG \in$ {\bf NP}, em que 
	\begin{center}
		$TRIANG = \{ \langle G \rangle$ | $G$ contém um triângulo$\}$.
	\end{center}

	\vspace*{0.3cm}
	
	{ \color{blue}
		{\bf Prova:} Se TRIANG $\in$ {\bf NP}, então é possível construir uma máquina de Turing não-determinística (MTN) que a decide em tempo polinomial. Construiremos a MTN $M$ que decide TRIANG:
		
		$MTN$ = ``Sobre a entrada $\langle G \rangle$, em que $G$ é um grafo não-direcionado, faça:
		\begin{enumerate}
			\item Selecione, de forma não-determinística, cada conjunto distinto $C$ com três vértices de $G$:
			\begin{enumerate}
				\item Verifique se $C$ forma um 3-clique em $G$.
				\item Se sim, {\it aceite}. 
				\item Caso contrário, {\it rejeite}''.
			\end{enumerate}
		\end{enumerate}
		
		O tempo de execução $t$ de $M$ é igual a soma do tempo de execução dos passos (a), (i), (ii), e (iii). Logo, $t = O(1) + O(1) + O(1) = O(1) = O(n^0)$. 
		
		0 é um número natural e TRIANG $\in$ {\sc NTIME}$(n^0)$. Logo, podemos afirmar que TRIANG $\in$ {\bf NP} $\blacksquare$
	}
	
	

\end{enumerate}

\section*{Teoremas Auxiliares}

\begin{itemize}
	
	\item[] {\bf Definição 1.16:} Uma linguagem é chamada de uma linguagem regular se algum autômato finito a reconhece.
	\item[] {\bf Teorema 1.25:} A classe de linguagens regulares é fechada sob a operação de união.
	\item[] {\bf Teorema 1.26:} A classe de linguagens regulares é fechada sob a operação de concatenação.
	\item[] {\bf Teorema 1.26.1:} A classe de linguagens regulares é fechada sob a operação de complemento.
	\item[] {\bf Teorema 1.39:} Todo autômato finito não-determinístico tem um autômato finito determinístico
	equivalente.
	\item[] {\bf Teorema 1.49:} A classe de linguagens regulares é fechada sob a operação estrela.
	\item[] {\bf Teorema 1.49.1:} A classe de linguagens regulares é fechada sob a operação de intersecção.
	\item[] {\bf Teorema 1.54:} Uma linguagem é regular se e somente se alguma expressão regular a descreve.
	\item[] {\bf Definição 3.5:} Chame uma linguagem de Turing-reconhecível se alguma máquina de Turing a reconhece.
	\item[] {\bf Definição 3.6:} Chame uma linguagem de Turing-decidível ou simplesmente decidível se alguma máquina de Turing a decide.
	\item[] {\bf Teorema 3.13:} Toda máquina de Turing multifita tem uma máquina de Turing que lhe é equivalente.
	\item[] {\bf Teorema 3.16:} Toda máquina de Turing não-determinística tem uma máquina de Turing determinística que lhe é equivalente.
	\item[] {\bf Teorema 3.21:} Uma linguagem é Turing-reconhecível se e somente se algum enumerador a enumera.
	\item[] {\bf Teorema 4.1:} $A_{AFD}$ é uma linguagem decidível.
	\item[] {\bf Teorema 4.2:} $A_{AFN}$ é uma linguagem decidível.
	\item[] {\bf Teorema 4.3:} $A_{EXR}$ é uma linguagem decidível.
	\item[] {\bf Teorema 4.4:} $V_{AFD}$ é uma linguagem decidível.
	\item[] {\bf Teorema 4.5:} $EQ_{AFD}$ é uma linguagem decidível.
	\item[] {\bf Teorema 4.9:} Toda linguagem livre-de-contexto é decidível.
	\item[] {\bf Teorema 4.11:} $A_{MT}$ é uma linguagem indecidível.
	\item[] {\bf Definição 4.14:} Um conjunto $A$ é contável se é finito ou tem o mesmo tamanho que $N$.
	\item[] {\bf Teorema 7.8: }
	Seja $t(n)$ uma função, em que $t(n) \geq n$. Então toda máquina de Turing multifita de tempo $t(n)$ tem uma máquina de Turing de um única fita equivalente de tempo $O(t^2(n))$.
	\item[] {\bf Teorema 7.11: }
	Seja $t(n)$ uma função, em que $t(n) \geq n$. Então toda máquina de Turing não-determinística de uma única fita de tempo $t(n)$ tem uma máquina de Turing de um única fita equivalente de tempo $2^{O(t(n))}$.	
	\item[] {\bf Definição 7.12: }
	{\bf P} é a classe de linguagens que são decidíveis em tempo polinomial sobre uma máquina de Turing determinística de uma única fita. Em outras palavras, {\bf P} = $\bigcup\limits_{k}$ {\bf TIME ($n^k$)}.
	\item[] {\bf Definição 7.19: }
	{\bf NP} é a classe das linguagens que têm verificadores de tempo polinomial.
	\item[] {\bf Teorema 7.20: }
	Uma linguagem está em {\bf NP} sse ela é decidida por alguma máquina de Turing não-determinística de tempo polinomial.  Em outras palavras, {\bf NP} = $\bigcup\limits_{k}$ {\bf NTIME ($n^k$)}.
\end{itemize}

\end{document}