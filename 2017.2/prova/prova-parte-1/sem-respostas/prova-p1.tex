\documentclass[12pt,a4paper,oneside]{article}

\usepackage[utf8]{inputenc}
\usepackage[portuguese]{babel}
\usepackage[T1]{fontenc}
\usepackage{amsmath}
\usepackage{amsfonts}
\usepackage{amssymb}
\usepackage{graphicx}
\usepackage{xcolor}
% Definindo novas cores
\definecolor{verde}{rgb}{0.25,0.5,0.35}

\author{\\Universidade Federal de Goiás (UFG) - Regional  Jataí\\Bacharelado em Ciência da Computação \\Teoria da Computação \\Esdras Lins Bispo Jr.}

\date{22 de fevereiro de 2018}

\title{\sc \huge Prova (Parte 1)}

\begin{document}

\maketitle

{\bf ORIENTAÇÕES PARA A RESOLUÇÃO}

\small
 
\begin{itemize}
	\item A avaliação é individual, sem consulta;
	\item A pontuação máxima desta avaliação é 10,0 (dez) pontos, sendo uma das 06 (seis) componentes que formarão a média final da disciplina: quatro testes, uma prova e exercícios;
	\item A média final ($MF$) será calculada assim como se segue
	\begin{eqnarray}
		MF & = & MIN(10, S) \nonumber \\
		S & = & (\sum_{i=1}^{4} 0,2.T_i ) + 0,2.P  + EB\nonumber
	\end{eqnarray}
	em que 
	\begin{itemize}
		\item $S$ é o somatório da pontuação de todas as avaliações,
		\item $T_i$ é a pontuação obtida no teste $i$,
		\item $P$ é a pontuação obtida na prova, e
		\item $EB$ é a pontuação total dos exercícios-bônus.
	\end{itemize}
	\item O conteúdo exigido desta avaliação compreende o seguinte ponto apresentado no Plano de Ensino da disciplina: (1) Teoria da Computação, (2) Modelos de Computação, e (3) Problemas Decidíveis.
\end{itemize}

\begin{center}
	\fbox{\large Nome: \hspace{10cm}}
\end{center}

\newpage

\begin{enumerate}
	
	\section{Primeiro Teste}
	
	\item (5,0 pt)	{\bf [Sipser 3.5]} Apresentamos logo abaixo a definição formal de uma máquina de Turing:
	
	\begin{center}
		\line(1,0){250}
	\end{center}	
	
	Uma {\bf máquina de Turing} é uma 7-upla $(Q, \Sigma, \Gamma, \delta, q_0, q_{aceita}, q_{rejeita})$, de forma que $Q, \Sigma, \Gamma$ são todos conjuntos finitos e
	
	\begin{itemize}
		\item $Q$ é o conjunto de estados,
		\item $\Sigma$ é o alfabeto de entrada sem o {\bf símbolo branco} $\sqcup$,
		\item $\Gamma$ é o alfabeto da fita, em que $\sqcup \in \Gamma$ e $\Sigma \subseteq \Gamma$,
		\item $\delta : Q \times \Gamma \rightarrow Q \times \Gamma \times \{E, D\}$ é a função de transição,
		\item $q_0 \in Q$ é o estado inicial,
		\item $q_{aceita} \in Q$ é o estado de aceitação, e
		\item $q_{rejeita} \in Q$ é o estado de rejeição, em que $q_{rejeita} \not= q_{aceita}$.
	\end{itemize}
	
	\begin{center}
		\line(1,0){250}
	\end{center}		
	
	Responda às seguintes perguntas, justificando a sua resposta.
	\begin{enumerate}
		\item (1,0 pt) Uma máquina de Turing pode alguma vez escrever o símbolo branco $\sqcup$ em sua fita?
		\item (1,5 pt) O alfabeto da fita $\Gamma$ pode ser o mesmo que o alfabeto de entrada $\Sigma$?
		\item (1,0 pt) A cabeça de uma máquina de Turing pode alguma vez estar na mesma localização em dois passos sucessivos?
		\item (1,5 pt) Uma máquina de Turing pode conter apenas um único estado?
	\end{enumerate}

	\item (5,0 pt) {\bf [Sipser 3.15 (d)]} Mostre que a coleção de linguagens decidíveis é fechada sob a operação de concatenação.
	
	\section*{Segundo Teste}
	
	\item (5,0 pt)  {\bf [Sipser 3.8 (b)]} Dê a descrição, em nível de implementação, da MT que decide a linguagem $A = \{\omega$ | $\omega$ contém duas vezes mais 0s que 1s$\}$. Admita que o alfabeto é o conjunto $\{0,1\}$.

	\item (5,0 pt) {\bf [Sipser 3.16 Adaptação]} Mostre que a coleção de linguagens Turing-reconhecíveis é fechada sob a operação de diferença \\(Dica: talvez seja útil saber que $A \setminus B = A \cap \overline{B})$.
	
	

\end{enumerate}

\end{document}