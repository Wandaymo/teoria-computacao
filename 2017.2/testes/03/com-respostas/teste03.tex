\documentclass[12pt,a4paper,oneside]{article}

\usepackage[utf8]{inputenc}
\usepackage[portuguese]{babel}
\usepackage[T1]{fontenc}
\usepackage{amsmath}
\usepackage{amsfonts}
\usepackage{amssymb}
\usepackage{graphicx}
\usepackage{xcolor}
% Definindo novas cores
\definecolor{verde}{rgb}{0.25,0.5,0.35}

\author{\\Universidade Federal de Goiás (UFG) - Regional  Jataí\\Bacharelado em Ciência da Computação \\Teoria da Computação \\Esdras Lins Bispo Jr.}

\date{25 de janeiro de 2018}

\title{\sc \huge Terceiro Teste}

\begin{document}

\maketitle

{\bf ORIENTAÇÕES PARA A RESOLUÇÃO}

\small
 
\begin{itemize}
	\item A avaliação é individual, sem consulta;
	\item A pontuação máxima desta avaliação é 10,0 (dez) pontos, sendo uma das 06 (seis) componentes que formarão a média final da disciplina: quatro testes, uma prova e exercícios;
	\item A média final ($MF$) será calculada assim como se segue
	\begin{eqnarray}
		MF & = & MIN(10, S) \nonumber \\
		S & = & (\sum_{i=1}^{4} 0,2.T_i ) + 0,2.P  + EB\nonumber
	\end{eqnarray}
	em que 
	\begin{itemize}
		\item $S$ é o somatório da pontuação de todas as avaliações,
		\item $T_i$ é a pontuação obtida no teste $i$,
		\item $P$ é a pontuação obtida na prova, e
		\item $EB$ é a pontuação total dos exercícios-bônus.
	\end{itemize}
	\item O conteúdo exigido desta avaliação compreende o seguinte ponto apresentado no Plano de Ensino da disciplina: (3) Problemas Decidíveis e (4) Problemas indecidíveis.
\end{itemize}

\begin{center}
	\fbox{\large Nome: \hspace{10cm}}
\end{center}

\newpage

\begin{enumerate}
	
	\section*{Terceiro Teste}
	
	\item (5,0 pt)  {\bf [Sipser 4.11]}  Seja $A = \{ \langle M \rangle$ | $M$ é um AFD que não aceita nenhuma cadeia contendo um número ímpar de 1s$\}$. Mostre que $A$ é decidível.
	
	\vspace{0.3cm}
	
	{\color{blue}
		R - É possível criar o AFD B de forma que $L(B) = \{\omega$ | $\omega$ tem um número ímpar de 1s$\}$ (porque $L(B)$ é regular). É necessário  verificar se $L(M) \cap L(B) = \emptyset$. Isto é possível, pois é possível construir o AFD C de forma que $L(C) = L(M) \cap L(B)$ (Teorema 1.49.1) e testar se $\langle C \rangle$ é membro de $V_{AFD}$ (Teorema 4.4). 
		
		Diante disto, será construído a seguir um decisor $M_A$ para $A$ :
		
		$M_A$ = ``Sobre a entrada $\langle M \rangle$, em que $M$ é um AFD, faça:
		\begin{enumerate}
			\item Construa os AFDs $B$ e $C$ conforme descritos anteriormente;
			\item Construa a MT $X$ que decide $V_{AFD}$ (Teorema 4.4);
			\item Rode $X$ sobre $\langle C \rangle$;
			\begin{enumerate}
				\item Se $X$ aceita, {\it aceite};
				\item Caso contrário, {\it rejeite}.
			\end{enumerate}					
		\end{enumerate}
		
		A linguagem $A$ é decidível pois foi possível construir uma máquina de Turing que a decide (Definição 3.6) $\blacksquare$
		
	}

	\item (5,0 pt) Seja $\mathcal{C}$ o conjunto de todas as sequências infinitas sobre os símbolos $\{a, b, c\}$. Mostre que $\mathcal{C}$ é incontável, usando uma prova por diagonalização.
	
	\vspace{0.3cm}
	
	{\color{blue}
		R - Vamos supor, por um momento, que $\mathcal{C}$ seja contável. Sendo contável, seja $f:\mathbb{N} \rightarrow \mathcal{C}$ a suposta bijeção existente entre $\mathbb{N}$ e $\mathcal{C}$ (tendo em vista que $\mathcal{C}$ não é finito - Definição 4.14). Logo, todos os elementos de $\mathcal{C}$ deveriam participar de $f$. Entretanto, é possível construir $s \in \mathcal{C}$, a partir de $f$, que não participa da suposta bijeção.
		
		Seja $g(n,d)$ a função que retorna o $d$-ésimo dígito da sequência infinita $f(n)$. Sejam $x,y \in \{a,b,c\} $ dois dígitos quaisquer. Definimos como $x \circ y$ a concatenação dos dígitos $x$ e $y$, nesta ordem. Também definimos $next(x)$ da forma como se segue
			$$next(x) = \begin{cases}
				b	&	\mbox{se } x=a, \\
				a 	&	\mbox{caso contrário}.
			\end{cases}$$.
		
		Logo, é possível construir a sequência infinita $s$ da seguinte forma:
		\begin{center}
			$s = next(g(1,1))\circ next(g(2,2)) \circ next(g(3,3)) \circ next(g(4,4)) \ldots$
		\end{center}
		
		Como é possível construir $s$, concluímos que a suposta bijeção $f$ não existe. Logo $\mathcal{C}$ é incontável $\blacksquare$
	}	

\end{enumerate}

\section*{Teoremas Auxiliares}

\begin{itemize}
	
	\item[] {\bf Definição 1.16:} Uma linguagem é chamada de uma linguagem regular se algum autômato finito a reconhece.
	\item[] {\bf Teorema 1.25:} A classe de linguagens regulares é fechada sob a operação de união.
	\item[] {\bf Teorema 1.26:} A classe de linguagens regulares é fechada sob a operação de concatenação.
	\item[] {\bf Teorema 1.26.1:} A classe de linguagens regulares é fechada sob a operação de complemento.
	\item[] {\bf Teorema 1.39:} Todo autômato finito não-determinístico tem um autômato finito determinístico
	equivalente.
	\item[] {\bf Teorema 1.49:} A classe de linguagens regulares é fechada sob a operação estrela.
	\item[] {\bf Teorema 1.49.1:} A classe de linguagens regulares é fechada sob a operação de intersecção.
	\item[] {\bf Teorema 1.54:} Uma linguagem é regular se e somente se alguma expressão regular a descreve.
	\item[] {\bf Definição 3.5:} Chame uma linguagem de Turing-reconhecível se alguma máquina de Turing a reconhece.
	\item[] {\bf Definição 3.6:} Chame uma linguagem de Turing-decidível ou simplesmente decidível se alguma máquina de Turing a decide.
	\item[] {\bf Teorema 3.13:} Toda máquina de Turing multifita tem uma máquina de Turing que lhe é equivalente.
	\item[] {\bf Teorema 3.16:} Toda máquina de Turing não-determinística tem uma máquina de Turing determinística que lhe é equivalente.
	\item[] {\bf Teorema 3.21:} Uma linguagem é Turing-reconhecível se e somente se algum enumerador a enumera.
	\item[] {\bf Teorema 4.1:} $A_{AFD}$ é uma linguagem decidível.
	\item[] {\bf Teorema 4.2:} $A_{AFN}$ é uma linguagem decidível.
	\item[] {\bf Teorema 4.3:} $A_{EXR}$ é uma linguagem decidível.
	\item[] {\bf Teorema 4.4:} $V_{AFD}$ é uma linguagem decidível.
	\item[] {\bf Teorema 4.5:} $EQ_{AFD}$ é uma linguagem decidível.
	\item[] {\bf Teorema 4.9:} Toda linguagem livre-de-contexto é decidível.
	\item[] {\bf Teorema 4.11:} $A_{MT}$ é uma linguagem indecidível.
	\item[] {\bf Definição 4.14:} Um conjunto $A$ é contável se é finito ou tem o mesmo tamanho que $N$.
\end{itemize}

\end{document}